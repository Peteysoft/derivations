\documentclass[11pt]{article}

\begin{document}

{\LARGE Calculating the incident fraction of solar radiation as a function of time and latitude}

First, consider a Cartesian coordinate system with the origin as the Earth's
centre and the $x-y$ plane lying along the equator.  If we model the Earth
as a sphere of radius $r$, a line of latitude will trace out a circle
given by the following equations:
\begin{eqnarray}
x & = & r \cos \phi \cos \theta \\
y & = & r \cos \phi \sin \theta \\
z & = & r \sin \phi
\end{eqnarray}
where $\phi$ is latitude and $\theta$ is an angle parameter.  If one is
standing at a given longitude, $\theta$ could also be considered the time
of day represented as an angle with $\theta=0$ representing noon, while
$\theta=\pi$ is midnight.

Now consider a new coordinate system with the x-y plane along the plane
of the ecliptic and the x-axis pointing towards the sun.  During the
winter solstice, the direction of the Earth's axis in the 
x-z plane will be described
by the following unit vector:
\begin{eqnarray}
x & = & \sin \gamma \\
z & = & \cos \gamma
\end{eqnarray}
where $\gamma$ is the angle of the Earth's tilt.
For a different time of year, which we represent as another angle, 
$\alpha$, ($\alpha=0$ is the winter solstice) 
we can project this vector onto the x-z plane as follows:
\begin{eqnarray}
x & = & \cos \alpha \sin \gamma \\
z & = & \cos \gamma
\end{eqnarray}
This will be the effective tilt of the Earth--
we are not concerned with the projection along the y-axis since this
has no effect on the calculations.

Now consider a ray of sunlight in our previous coordinate system.
By rotating the effective axis by ninety degrees and normalizing it
we get the direction vector pointing to the ray's source (i.e., the sun).
\begin{equation}
\hat v = \frac{(\cos \gamma, ~ 0, ~ -\cos \alpha \sin \alpha)}
		{\sqrt{\cos^2 \alpha \sin^2 \gamma + \cos^2 \gamma}}
\end{equation}
To get the zenith angle of the Sun, we simply take the dot-product
of this with the normal vector:
\begin{eqnarray}
\hat n & = & (\cos \phi \cos \theta, ~ \cos \phi \sin \theta, ~ \sin \phi) \\
\beta & = & \cos^{-1} (\hat v \cdot \hat n) \\
& = & \cos^{-1}\left ( \frac{\cos \phi \cos \theta \cos \gamma -
	\sin \phi \cos \alpha \sin \gamma} 
	{\sqrt{\cos^2 \alpha \sin^2 \gamma + \cos^2 \gamma}} \right )
\end{eqnarray}
where $\beta$ is the local solar zenith angle and $\hat n$ is the normal vector.  
The cosine of this angle
gives the incident fraction of
solar radiance in the absence of any atmosphere for all daylight
values, that is those for which it is greater than one or $\beta < \pi/2$.

For use in ocean and similar simulations, we generally don't care about
diurnal variations so this result would be too highly resolved.
A more useful value would be the fraction of incident solar radiation
averaged over all hours of daylight.  The daylight values of $\theta$
are given as:
\begin{eqnarray}
\hat v \cdot \hat n & > & 0 \\
\frac{\cos \phi \cos \theta \cos \gamma - \sin \phi \cos \alpha \sin \gamma}
	{\sqrt{\cos^2 \alpha \sin^2 \gamma + \cos^2 \gamma}} & > & 0 \\
\cos \theta & > & \frac{\sin \phi \cos \alpha \sin \gamma}{\cos \phi \cos \gamma}\\
| \theta | & < & \cos^{-1} (\tan \phi \cos \alpha \tan \alpha) 
\end{eqnarray}
Now we integrate $\cos \beta$ over these values of $\theta$:
\begin{eqnarray}
\theta_0 & = & \cos^{-1} (\tan \phi \cos \alpha \tan \alpha) \\
\bar Q /Q_0 & = & \frac{1}{2\pi} \int_{-\theta_0}^{\theta_0} \hat v \cdot \hat n \\
	& = & \frac{1}{\pi} \int_0^{\theta_0} 
	\frac{\cos \phi \cos \theta \cos \gamma - \sin \phi \cos \alpha \sin \gamma}
	{\sqrt{\cos^2 \alpha \sin^2 \gamma + \cos^2 \gamma}} d\theta \\
	& = & \frac{\cos \phi \cos \gamma \sin \theta_0 - \theta_0 \sin \phi \cos \alpha \sin \gamma}
		{\pi \sqrt{\cos^2 \alpha \sin^2 \gamma + \cos^2 \gamma}} 
\end{eqnarray}
where $\bar Q$ is the average amount of solar radiation incident on the Earth's surface as a
function of latitude and time of year while $Q_0$ is the total amount of solar radiation
travelling through space--the so-called ``solar constant.''  

Note that for high latitudes
and times of year close to the solstice, there will be values
for which the $\theta_0$ will be undefined because the cosine from which it is calculated
is either greater than or less than one.  In this case we set it to the extremum of its
allowable range, that is the cosine goes to either to one or
minus one respectively giving values for $\theta_0$ of $0$ and $\pi$.
In other words, for the purpose of this integration,
the sun cannot rise after noon and set before noon, nor
can it rise before midnight and set after midnight because this would give
total hours of daylight less than zero or greater than twenty-four hours.

All of this of course neglects any atmospheric effects such as diffraction,
scattering or absorption.

 
\end{document}
 
