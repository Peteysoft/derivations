\documentclass[12pt]{article}

\usepackage{amsmath}
\usepackage[dvips]{graphicx}
\usepackage{natbib}

\bibliographystyle{apa}

\begin{document}

{\Large The mathematics of tracer transport}

\section{Fundamentals}

A trajectory is the motion in time of an infinitesimal packet of fluid
as it is carried along by the flow:
\begin{equation}
\frac{\mathrm d \vec x}{\mathrm d t}=\vec v(\vec x,~t)
\end{equation}
where $\vec x$ is position, $t$ is time and $\vec v$ is the flow field.
If we integrate this in time, starting at $t_0$ and ending at $t=t_0+\Delta t$, we get a
vector function, call it, $\Phi$:
\begin{equation}
\vec x=\Phi(\vec x_0,~t_0,~\Delta t)
\label{traj_def}
\end{equation}
where $x_0=x(t_0)$ is the Lagrangian or starting coordinate.
Unlike in \citet{Ottino1989}, we specify the starting time explicitly instead 
of assuming that $t_0=0$.  This convention will become useful later on, e.g., 
when chaining functions:
\begin{equation}
\vec x=\Phi(\vec x_0,~\Delta t_1+\Delta t_2)=\Phi[\Phi(\vec x_0,~t_0,~\Delta t_1),~t_0 + \Delta t_1,~\Delta t_2]
\end{equation}

A flow tracer is a scalar field that follows the flow, typically a dissolved
trace substance, but could also comprise a suspension or in fact any property
of the fluid such as temperature or the velocity field itself.

The change in the total amount of tracer in a fixed volume, $V$, is given
by the flux plus the source term:
\begin{equation}
\frac{\partial}{\partial t}\int_V \rho \mathrm d \vec x=-\oint_A \rho \vec v \cdot \hat n \mathrm d A
	+ \int_V S \mathrm d \vec x
\end{equation}
where $\rho$ is the density of the tracer,
$A$ is the area enclosing $V$ and $S$ is the source term, which for the moment,
we will take to be zero but in later analysis will 
be needed for the diffusion term.
From Gauss's law:
\begin{equation}
\int_V \frac{\partial \rho}{\partial t} \mathrm d \vec x=-\int_V \nabla \cdot (\rho \vec v)\mathrm d \vec x 
\end{equation}
Removing the integrals and expanding the first term on the right side:
\begin{equation}
\frac{\partial \rho}{\partial t} =-\vec v \cdot \nabla \rho - \rho \nabla \cdot \vec v 
\label{mass_conservation_Eulerian}
\end{equation}
The first term is the advection term while the second term is the mass
conservation term and is governed by the 
expansion and contraction of the fluid.  In an incompressible fluid (that is,
$\nabla \cdot \vec v = 0$), this term will be zero.

To go from the Lagrangian to the Eulerian formulation, 
we use the following equation for the {\it total derivative}:
\begin{equation}
\frac{\mathrm d \rho}{\mathrm d t}
 = \frac{\partial \rho}{\partial t} + \vec v \cdot \nabla \rho
\end{equation}
Combining this with Equation (\ref{mass_conservation_Eulerian}):
\begin{equation}
\frac{\mathrm d \vec \rho}{\mathrm d t} =  - \rho \nabla \cdot \vec v
\label{mass_conservation_Lagrangian}
\end{equation}
This is the Lagrangian equation for conservation of mass and governs the total density
of the fluid, that is, fluid density is itself a flow tracer.

Suppose we use a {\it mixing ratio} instead of density to track the
tracer:
\begin{equation}
q = \frac {\rho}{\rho_t}
\end{equation}
where $\rho_t$ is the total density of the fluid.  Differentiating this wrt
$t$ and then substituting Equation (\ref{mass_conservation_Lagrangian})
produces the following:
\begin{eqnarray}
\frac{\mathrm d q}{\mathrm d t} & = & \frac{1}{\rho_t} \frac{\mathrm d \rho}{\mathrm d t}
	- \frac{\rho}{\rho_t^2}\frac{\mathrm d \rho_t}{\mathrm d t}\\
& = & 0
\end{eqnarray}
or, in Eulerian form:
\begin{equation}
\frac{\partial q}{\partial t} = - \vec v \cdot \nabla q
\end{equation}
This is the {\it advection equation}.

\section{Volume deformation}

\label{deformation_section}

Much of this analysis can be found in \citet{Pattanayak2001} and 
\citet{Mills2004}, however we repeat it here since I seem to end up re-deriving
it roughly every eight months.

The instantaneous rate of stretching of Lagrangian space is given by the
gradient of the velocity, $\nabla \vec v$.
Suppose we perturb a trajectory by a minute amount, $\delta \vec x$.
The Tayler expansion of the time derivative is, to first order:
\begin{equation}
\frac{\mathrm d}{\mathrm d t} (\vec x + \delta \vec x) \approx
	\vec v + \nabla \vec v \cdot \delta \vec x
\end{equation}
or:
\begin{equation}
\frac{\mathrm d}{\mathrm d t}\delta \vec x \approx \nabla \vec v \cdot \delta \vec x
\label{evolution_error_vector}
\end{equation}
Whether we left multiply or right multiply depends on which convention we adopt
for the application of the gradient or nabla operator, $\nabla$, to a vector,
therefore we write it out component-by-component:
\begin{equation}
\nabla \vec v = \left [
\begin{array}{ccc}
\frac{\partial v_x}{\partial x} & \frac{\partial \vec v_x}{\partial y} & \frac{\partial \vec v_x}{\partial z} \\
\frac{\partial v_y}{\partial x} & \frac{\partial \vec v_y}{\partial y} & \frac{\partial \vec v_y}{\partial z} \\
\frac{\partial v_z}{\partial x} & \frac{\partial \vec v_z}{\partial y} & \frac{\partial \vec v_z}{\partial z}
\end{array} \right ]
\end{equation}
where $\vec x=[x_1,~x_2,~x_3]=[x,~y,~z]$.

We define $H(\vec x_0,~t_0,~\Delta t)$ as follows:
\begin{eqnarray}
\frac{\mathrm d H}{\mathrm d t} & = & \nabla \vec v \cdot H \\
\label{deformation_matrix}
H(\vec x_0,~t_0,~0) & = & I
\end{eqnarray}
where $I$ is the identity matrix.
This is known as the {\it tangent model} of a dynamical system and applied
to a trajectory is the total stretching of Lagrangian space.

We can relate $H$ to the integrated trajectory, $\Phi$, as follows:
\begin{eqnarray}
\frac{\mathrm d \Phi}{\mathrm d t} & = & v(\Phi, ~t) \\
\frac{\mathrm d}{\mathrm d t} 
\nabla_{\vec x_0} \Phi & = & \nabla \vec v \cdot \nabla_{\vec x_0} \Phi
\end{eqnarray}
or,
\begin{equation}
\nabla_{\vec x_0} \Phi = H
\label{advection_eqn}
\end{equation}

In conjunction with the equations for the evolution of error vectors, above,
we can derive a set of corresponding equations for evolution of the tracer
gradient.  Taking the gradient of Equation (\ref{advection_eqn}):
\begin{equation}
\frac{\partial}{\partial t} \nabla q = -\nabla q \cdot \nabla \vec v -
		\vec v \cdot \nabla \nabla q 
\end{equation}
Meanwhile, using the total derivative:
\begin{equation}
\frac{\mathrm d}{\mathrm d t} \nabla q = \vec v \cdot \nabla \nabla q + 
		\frac{\partial}{\partial t} \nabla q
\end{equation}
Combining the two:
\begin{equation}
\frac{\mathrm d}{\mathrm d t} \nabla q = -\nabla q \cdot \nabla \vec v
\label{evolution_tracer_gradient}
\end{equation}

In parallel with $H$, we define $H^\prime$:
\begin{eqnarray}
\frac{\mathrm d H^\prime}{\mathrm d t} & = & -H^\prime \cdot \nabla \vec v\\
\label{inverse_deformation_matrix}
H^\prime(\vec x_0,~t_0,~0) & = & I
\end{eqnarray}
It is easy to show that:
\begin{equation}
H^{-1}=H^\prime
\end{equation}
by taking the time derivative of $H^\prime \cdot H$:
\begin{eqnarray}
\frac{\mathrm d}{\mathrm d t} H^\prime \cdot H & = & 
		H^\prime \cdot \frac{\mathrm d H}{\mathrm d t} +
		\frac{\mathrm d H^\prime}{\mathrm d t} \cdot H\\
		& = & H^\prime \cdot \nabla \vec v \cdot H 
		- H^\prime \cdot \nabla \vec v \cdot H \\
		& = & 0
\end{eqnarray}
and that:
\begin{equation}
\nabla q = \nabla q_0 \cdot H^\prime
\end{equation}

Define $\Phi^{-1}(\vec x,~t,~\Delta t)$ as:
\begin{equation}
\Phi^{-1}[\Phi(\vec x_0,~t_0,~\Delta t),~t_0+\Delta t,~\Delta t]=\vec x_0
\end{equation}
Thus,
\begin{equation}
H^\prime=\nabla \Phi^{-1}
\end{equation}

\section{Transport map}

\label{map_section}

The dynamics of the tracer, $q$, can be summarized using a linear
{\it tranpsort map}, $Q$:
\begin{equation}
q(\vec x,~t)=\int_V Q(\vec x_0,~\vec x,~t_0,~\Delta t) q(\vec x_0,~t_0) \mathrm d \vec x_0
\label{tracer_map_continuous}
\end{equation}
In the absence of diffusion or source terms, this map can be calculated in
at least two ways, by a differential equation: 
\begin{eqnarray}
\frac{\partial}{\partial t} Q & = & (\vec v \cdot \nabla) Q 
\label{tracer_map_continuous1}\\
Q(\vec x_0,~\vec x,~t_0,~0) & = & \delta(\vec x-\vec x_0) 
\label{tracer_map_continuous2}
\end{eqnarray}
where $\delta$ is the Dirac delta function, and using the trajectory function:
\begin{equation}
Q(\vec x_0,~\vec x,~t_0,~\Delta t) = \delta[\Phi(\vec x_0,~t_0,~\Delta t)-x_0]
\end{equation}
In the latter case, combination with (\ref{tracer_map_continuous}) produces the
Frobenius-Perron equation. \citep{Ott1993}

The tracer map can be approximated by a matrix:
\begin{equation}
\vec q(t) = R \cdot \vec q(t_0)
\end{equation}
where $\vec q$ is a discrete approximation of the tracer configuration:
$q_i = q(\vec x_i)$.  Tracer transport is fully linear and the time evolution
of $R$ is governed by the same mathematics as $H$ and $H^\prime$:
\begin{equation}
\frac{\mathrm d R}{\mathrm d t} = A \cdot R
\label{discrete_tracer_map}
\end{equation}
There are a number of ways to calculate $A$, the most obvious being from an 
Eulerian finite difference scheme, which, in one dimension, would look something
like this:
\begin{equation}
\frac{\mathrm d r_{ij}}{\mathrm d t} = \frac{v(x_j,~t)}{\Delta x_{j-1}+\Delta x_j} (r_{i,j+1} - r_{i,j-1})
\label{simple_finite_difference}
\end{equation}
or,
\begin{eqnarray}
a_{i-1,i} & = & - \frac{v(x_i,~t)}{\Delta x_{i-1}+\Delta x_i} \\
a_{i+1,i} & = & \frac{v(x_i,~t)}{\Delta x_{i-1}+\Delta x_i}
\end{eqnarray}
Neither representation shows the {\it boundary conditons}.  Typically, $A$
will be either band diagonal or, as in the case of a semi-Lagrangian
scheme, quite close.

For simplicity, we will analysize both this system and the pair of systems
discussed in the previous section using matrix algebra.  A more thorough
treatment would use abstract algebra and operator theory, especially for
the tracer map.

\section{Matrix analysis of systems of linear ODEs} 

\subsection{Analytic solution of the stationary case}

We wish to solve the system of linear ordinary differential equations (ODEs):
\begin{equation}
\frac{\mathrm d \vec r}{\mathrm d t}=A \cdot \vec r
\label{linear_ODE_system_vector_soln}
\end{equation}
Supposing $A$ has no time dependence, we perform an eigenvector 
decomposion:
\begin{equation}
A = V^{-1} \Lambda V
\end{equation}
where $V$ is a matrix of eigenvectors, and $\Lambda$ is a diagonal matrix
of eigenvalues, $\lambda_{ii}=\lambda_i$.  Thus,
\begin{equation}
\frac{\mathrm d}{\mathrm d t} V \cdot \vec r=\Lambda \cdot V \cdot  \vec r
\end{equation}
By performing the linear coordinate transformation,
\begin{equation}
\vec r^\prime=V \cdot \vec r
\end{equation}
the equation is easily solved:
\begin{equation}
r^\prime_i = e^{\lambda_i \Delta t} r^\prime_i(t=t_0)
\end{equation}
or, in the un-transformed system:
\begin{eqnarray}
\vec r & = & \left [V^{-1} \cdot \exp(\Delta t \Lambda) \cdot V \right ] \cdot \vec r 
\label{solution_no_time_dependence} \\
& \equiv & \exp(\Delta t A)\cdot \vec r
\end{eqnarray}

\subsection{Solving the time-dependent case}

The stationary case is interesting, but what can it tell us about the 
more general case in which $A$ is time-dependent?
Here the problem is further generalized so that a matrix, $R$ 
is used in place of the vector, $\vec r$:
\begin{equation}
\frac{\mathrm d \vec R}{\mathrm d t}=A(t) \cdot R
\label{linear_ODE_system_matrix_soln}
\end{equation}
One of the most important properties of the solution is that it
is decomposable as follows:
\begin{equation}
R(0,~t_n) = R(t_{n-1},\,\Delta t_n) \cdot R(t_{n-2},\,\Delta t_{n-1}) \cdot \, ...~~ \cdot R(t_2,\, \Delta t_3) \cdot R(t_1,\,\Delta t_2) \cdot R(0,\,\Delta t_1)
\label{matrix_soln_decomposition}
\end{equation}
where $t_i=\sum_i \Delta t_i$ and we have used the convention, begun
in Equation (\ref{traj_def}) of making $R$ a function both of the
initial time and of the subsequent time interval.  It follows that $R(t_0, 0)=I$.

Going back to Equation (\ref{linear_ODE_system_vector_soln}), we can
see that the solution is given by a product of the initial vector with
the matrix solution:
\begin{equation}
\vec r(t_0+\Delta t)=R(t_0, \Delta t) \cdot \vec r(t_0)
\end{equation}

\subsection{Negative and left-multiplied cases}

Consider the case in which the order of the factors in on the R.H.S of Equation 
(\ref{linear_ODE_system_vector_soln}) are reversed (left-multiply vs. right-multiply):
\begin{equation}
\frac{\mathrm d \vec r}{\mathrm d t} = \vec r \cdot A
\end{equation}
This case is particularly important, since Equation (\ref{evolution_error_vector})
when transformed in this way becomes the vorticity equation.

We start with the analytic solution of the stationary case:
\begin{eqnarray}
A & = & V^{-1} \cdot \Lambda \cdot V \\
\frac{\mathrm d \vec r}{\mathrm d t} & = & \vec r \cdot V^{-1} \cdot \Lambda \cdot V \\
\frac{\mathrm d}{\mathrm d t} (\vec r \cdot V^{-1}) & = & (\vec r \cdot V^{-1}) \cdot \Lambda \\
\vec r \cdot V^{-1} & = & \vec r_0 \cdot V^{-1} \cdot \exp (\Lambda t) \\
\vec r & = & \vec r_0 \cdot V^{-1} \cdot \exp (\Lambda t) \cdot V 
\end{eqnarray}
In other words, the solution is the same, but the initial conditions are
left-multiplied instead of right multiplied, or equivalently, the whole thing 
could be transposed.

In the time-dependent case each element of the solution is in reverse order,
not just the initial conditions:
\begin{equation}
\vec r(t_n) = \vec r_0 \cdot R^*(t_0,\Delta t_0) \cdot R^*(t_1, \Delta t_1) \cdot ~ ... 
~ \cdot R^*(t_{n-2},\Delta t_{n-2}) \cdot R^*(t_{n-1},\Delta t_{n-1})
\end{equation}
where $R^*$ is a solution to the equation:
\begin{eqnarray}
\frac{\mathrm d}{\mathrm d t}R^*(t_0,~t) & = & R^*(t_0, ~t) \cdot A(t) \\
R^*(t_0, ~ 0) & = & I
\end{eqnarray}
and,
\begin{equation}
R^*(t,~\Delta t) \approx \exp \left [ A(t) \Delta t \right ]
\end{equation}
in the limit as $\Delta t \rightarrow \infty$.

For the case of a negative R.H.S.:
\begin{equation}
\frac{\mathrm d \vec r}{\mathrm d t} = - A \cdot \vec r
\end{equation}
it is easy to show that the stationary solution is simply inverted:
\begin{eqnarray}
\vec r & = & V^{-1} \cdot \exp (-\Lambda t) \cdot V \cdot \vec r \\
& = & V^{-1} \cdot \left [ \exp (\Lambda t) \right ]^{-1} \cdot V \cdot \vec r \\
& = & \left [V^{-1} \cdot \exp (\Lambda t) \cdot V \right]^{-1} \cdot \vec r 
\end{eqnarray}
while for the time-dependent case, we have:
\begin{equation}
\vec r \approx R^{-1}(t_{n-1},\Delta t_{n-1}) \cdot R^{-1}(t_{n-2}, \Delta t_{n-2}) \cdot ~ ... 
~ \cdot R^{-1}(t_1,\Delta t_1) \cdot R^{-1}(t_0,\Delta t_0) \cdot \vec r_0
\end{equation}
assuming that each $\Delta t_i$ is small enough that $R$ approximates the 
stationary case.
This provides an alternative derivation for the solution of the negative,
transposed (left-multiplied) case which we saw already in (\ref{evolution_tracer_gradient}):
\begin{equation}
\frac{\mathrm d R}{\mathrm d t} = -\vec r \cdot A
\end{equation}
which is given by:
\begin{eqnarray}
\vec r(t_n) & \approx & \vec r_0 \cdot R^{-1}(t_0,\Delta t_0) \cdot R^{-1}(t_1, \Delta t_1) \cdot ~ ... 
~ \cdot R^{-1}(t_{n-2},\Delta t_{n-2}) \cdot R^{-1}(t_{n-1},\Delta t_{n-1}) \\
& = & \vec r_0 \cdot R^{-1}(t_0, t_n-t_0)
\end{eqnarray}
Where $R$ is the solution to Equation (\ref{linear_ODE_system_matrix_soln}).

\subsection{Volume conservation}

Since each element in (\ref{matrix_soln_decomposition}) approaches the
non-time-dependent solution given by the first factor of 
(\ref{solution_no_time_dependence}) in the limit as $\Delta t_i\rightarrow 0$, 
we can use the properties of the non-time-dependent solution to
reason about those of the time-dependent one.

In many case of the problems discussed in Section \ref{deformation_section}
and Section \ref{map_section}, the coefficient matrix, $A$, will have
special properties that will affect the solution.
In the solution of (\ref{deformation_matrix}) for instance, 
$A=\nabla \vec v$ while the velocity field, $\vec v$, is frequently non-divergent,
that is $\nabla \cdot \vec v=0$ or $\mathrm{Tr}(A)=0$.
It can be shown (we won't show it here) that 
if the trace of a matrix is zero, then the eigenvalues of that matrix
sum to zero.

The determinant of the solution matrix,
$V^{-1}\cdot\exp(\Delta t\Lambda)\cdot V$, in (\ref{solution_no_time_dependence}) is:
\begin{eqnarray}
|V^{-1}||\exp(\Delta t\Lambda)||V| & = & \frac{1}{|V|}\prod_i \exp(\Delta t \lambda_i) |V| \\
& = & \exp\left(\Delta t \sum_i \lambda_i\right) \\
& = & 1
\end{eqnarray}
In other words, the solution, in this case, is {\it volume-conserving}:
volumes in the space, e.g. as calculated by the determinant of a set of 
solution vectors
spanning the space, are conserved.  This is known as Liousville Theorem.

\subsection{SVD and the Lyapunov spectrum}

The singular value decomposition of a matrix is given as:
\begin{equation}
R=U\cdot S\cdot V^T
\label{SVD_def}
\end{equation}
where $R$ is an $[m*n]$ matrix, $U$ is an $[m*n]$ ortho-normal
matrix, $S$ is an $[n*n]$ diagonal matrix of {\it singular values}
($s_{ii}=s_i>s_{i+1}$) and $V$ is an $[n*n]$ ortho-normal matrix.

$U$ and $V^T$ are also termed the left and right {\it singular vectors}, 
respectively and are normally calculated through eigenvalue analysis:
\begin{eqnarray}
R\cdot R^T \cdot U & = & U\cdot S^2 \\
V^T \cdot R^T \cdot R & = & S^2 \cdot V^T
\end{eqnarray}
Typically, only one of $U$ or $V$ is calculated and then the other by projection
onto $R$.  Which one is calculated first is best determined by whether $m$
is greater than or less than $n$.  Equation (\ref{SVD_def}) assumes that 
$m>n$.

Assuming that $R$ is an integrated tangent model as in 
(\ref{deformation_matrix}), (\ref{inverse_deformation_matrix})
and (\ref{discrete_tracer_map}), 
the Lyapunov exponents are defined as the time averages of the logarithms
of the singular values in the limit as time goes to infinity:
\begin{equation}
\lambda_i=\lim_{\Delta t \rightarrow 0} \frac{1}{\Delta t} \log s_i
\end{equation}

If the tangent model is volume conserving, then the Lyapunov exponents, like
the eigenvalues, will sum to zero.

\subsection{Other properties}

Suppose $R$ has the property that it preserves lengths when applied to
a vector:
\begin{equation}
|R\cdot \vec q| = |\vec q|
\end{equation}
thus the rate of change of the vector will always be perpendicular:
\begin{eqnarray}
\frac{\mathrm d}{\mathrm d t}|\vec q| & = & 
	\nabla |\vec q| \cdot \frac{\mathrm d \vec q}{\mathrm d t} \\
	&=& \frac{\vec q}{|\vec q|} \cdot A \cdot \vec q \\
	&=& 0 \\
	\vec q \cdot A \cdot \vec q & = & 0 
\end{eqnarray}
$A$ is what we shall call an ``orthogonality transform''.
By separating the diagonal and off-diagonal components in the last expression,
\begin{equation}
\sum_i \sum_j a_{ij} q_i q_j = \sum_{i=1}^n a_{ii}q_i^2 + \sum_{i=1}^n \sum_{j=i+1}^n (a_{ij} q_i q_j + a_{ji} q_i q_j) = 0
\end{equation}
we can show that it has the following
properties:
\begin{eqnarray}
a_{ii} & = & 0 \\
a_{ij}+a_{ji} & = & 0
\end{eqnarray}

Meanwhile, $R$ is a rotation.  All the singular values of $R$ will be 1:
\begin{eqnarray}
\vec q \cdot R^T \cdot R \cdot \vec q & = & \vec q \cdot \vec q\\
R^T \cdot R \cdot \vec q & = & \vec q
\end{eqnarray}
Because it is a rotation, it will be orthonormal with a determinant of 1
and ones for all the eigenvalues.

The tracer mapping, $Q$, as defined in 
(\ref{tracer_map_continuous1}) and (\ref{tracer_map_continuous2}),
that is, without diffusion, will fulfill these properties.  Discrete mappings
can only approximate them and by necessity always include some diffusion.

A more important property of tracer advection is that the total substance
remains constant, thus:
\begin{eqnarray}
\sum_i q_i & = & const. \\
\frac{\mathrm d}{\mathrm d t}\sum_i q_i & = & 0
\end{eqnarray}
This will be true even if $q$ is measured as a mixing ratio, provided that
the flow is non-divergent.  Most flows in real fluids are approximately 
non-divergent, especially when considered over long time scales.
Continuing the derivation:
\begin{eqnarray}
\sum_i \sum_j r_{ij} q_j & = & \sum_j q_j \\
\sum_j q_j \left ( \sum_i r_{ij} - 1 \right ) & = & 0 \\
\end{eqnarray}
and:
\begin{eqnarray}
\sum_i \frac{\mathrm d q_i}{\mathrm d t} & = & 0 \\
\sum_i \sum_j a_{ij} q_j & = & 0 \\
\sum_j q_j \sum_i a_{ij} & = & 0 
\end{eqnarray}
Therefore:
\begin{eqnarray}
\sum_i r_{ij} & = & 1 
\label{column_sums_to_one}\\
\sum_i a_{ij} & = & 0
\end{eqnarray}
[Problem: show that the second property implies the first using an eigenvalue
solution for an A with no time dependence.]

As pointed out already, a discrete tracer mapping will always require some 
amount of diffusion.  This means that the tracer configuration will 
tend towards a uniform distribution over time, 
that is, it will "flatten out."  We can
show that, given the above constraint, a tracer field with all the same values
 has the smallest magnitude.  Suppose there are only two elements in the 
tracer vector, $\vec q=\lbrace q,~q \rbrace$.  The magnitude of the vector is:
\begin{equation}
|\vec q|=\sqrt{q^2+q^2}=\sqrt{2} q^2
\end{equation}
Now we introduce a separation between the elements, $2\Delta q$, that 
nonetheless keeps the sum of the elements constant:
\begin{eqnarray}
|q+\Delta q,~q-\Delta q| & = & \sqrt{(q+\Delta q)^2+(q-\Delta q)^2} \\
& = & \sqrt{2}\sqrt{q^2+(\Delta q)^2} \ge \sqrt{2} q^2
\end{eqnarray}
This will generalize to higher-dimensional vectors.  In general, we can
say that:
\begin{equation}
\vec q \cdot R^T \cdot R \cdot \vec q \le \vec q \cdot \vec q
\label{tracer_map_inequality}
\end{equation}
Implying that for the eigenvalue problem,
\begin{eqnarray}
R^T \cdot R \cdot \vec q & = & s \vec q \nonumber\\
|s| & \le & 1 \label{SV_inequality}
\end{eqnarray}
otherwise (\ref{tracer_map_inequality}) would not hold.
Since the tracer field is always positive, $s$ will also be positive.
Equation (\ref{SV_inequality}) implies that the Lyapunov exponents are all
either zero or negative with the largest equal to 0.  This has been shown
numerically in \citet{Mills2012}.

\section{Practical considerations}

The traditional method of testing Eulerian tracer simulations for stability
is von Neumann analysis.
Equation (\ref{simple_finite_difference}) provides a simple, finite
difference method for
calculating the evolution matrix, $A$, for an Eulerian tracer simulation.
Assuming we've obeyed the Courant-Friedrich-Levy (CFL) criterion for choosing
the time-step (which itself is derived from von Neumann analysis), von Neumann
analysis shows this method to be unconditionally stable.

In fact the method is quite naive and if applied to any real fluid flow would 
quickly overflow.
This is because shear flows and mixing cause the tracer gradient to grow without bound.
To fix this, some amount of diffusion must be added.  Here is the 
{\it advection-diffusion} equation:
\begin{equation}
\frac{\partial q}{\partial t} = - \vec v \cdot \nabla q + \nabla \cdot D \cdot \nabla q
\end{equation}
where $D$ is the diffusivity tensor.
Here is its translation, in one dimension, to a second-order, centred,
finite-difference equation:
\begin{eqnarray}
\frac{\partial q_i}{\partial t} & = & \frac{v(q_{i+1} - q_{i-1})}{2 \Delta x} +
	\frac{d (q_{i-1} + q_{i+1} - 2 q_i}{\Delta x^2} \\
& = & \left (- \frac{v}{2 \Delta x} + \frac{d}{\Delta x^2} \right ) q_{i-1} -
	\frac{2 d}{\Delta x^2} q_i + 
	\left (\frac{v}{2 \Delta x} + \frac{d}{\Delta x^2} \right ) q_{i+1}
\end{eqnarray}
where $d$ is a scalar diffusion coefficient.
Expressed as elements of a matrix:
\begin{eqnarray}
a_{i,i-1} & = & \left (- \frac{v}{2 \Delta x} + \frac{d}{\Delta x^2} \right ) \\
a_{i,i} & = & \frac{2 d}{\Delta x^2} \\
a_{i,i+1} & = & \left (\frac{v}{2 \Delta x} + \frac{d}{\Delta x^2} \right )
\end{eqnarray}
In order to prevent the simulation from overflowing, $d$ will have to be tuned.
Generalization to two or more dimensions is straightforward.
Since the simulation is discrete, the actual order of these points in the
vector or matrix is arbitrary.

Semi-Lagrangian methods perform a back-trajectory from each of the Eulerian
points in the simulation, then interpolate the new value.  Instead of doing
this directly, the interpolation coefficients are stored directly, not in
the evolution matrix, $A$, but rather in a transport matrix, $R(t_i,~\Delta t_i)$.
Each Eulerian time-step produces one element in the decomposition of a
larger transport matrix $R(t_0,~t_n-t_0)$.

Semi-Lagrangian methods are unconditionally stable and not subject to the
CFL criterion.
Both the final transport map, and each member of its decomposition will have
the following properties:
\begin{itemize}
\item every element is between 0 and 1:
\begin{equation}
0 \le r_{ij} \le 1
\label{element_on_unit_interval}
\end{equation}
\item rows sum to 1:
\begin{equation}
\sum_j r_{ij} = 1
\end{equation}
\end{itemize}
In addition, we can approximate the evolution matrix:
\begin{equation}
A(t_i+\Delta t/2}) \approx \frac{1}{\Delta t} (R(t_i,~\Delta t) - I)
\end{equation}

\subsection{Criteria for stability}

If the transport map, $R$, satisfies both Equation (\ref{column_sums_to_one})
and the inequality in (\ref{element_on_unit_interval}) then it is mathematically
equivalent to a conditional probability.
In addition, if we normalize the tracer such that $\sum_i q_i=1$
then it too is equivalent to a probability.

The criterion for a diffusive simulation can be translated


\section{Adding sources and sinks}

We wish to add a source term, $\vec \sigma$, to the matrix tracer model
described in the previous two sections:
\begin{equation}
\frac{\mathrm d R}{\mathrm d t} = A(t) \cdot R + \vec \sigma(t)
\end{equation}
Integrating to a discrete approximation:
\begin{eqnarray}
\vec q(t_n) & = R(t_{n-1}, \Delta t_{n-1}) \cdot [\Delta t_{n-1}\vec \sigma(t_{n-1})+R(t_{n-1}, \Delta t_{n-2}) 
\cdot [\Delta t_{n-2} \vec \sigma(t_{n-2})+~...\nonumber \\
& R(t_2, \Delta t_2) \cdot [\Delta t_2 \vec \sigma(t_2) +
R(t_1, \Delta t_1) \cdot [\Delta t_1 \vec \sigma(t_1) + R(t_0, \Delta t_0) \cdot q(t_0) ]]...]]
\end{eqnarray}
and multiplying through:
\begin{eqnarray}
\vec q(t_n) & = & R(t_0,t_n-t_0) \cdot \vec q(t_0) + \Delta t_1 R(t_1,t_n-t_1) \cdot \vec \sigma(t_1) +
\Delta t_2 R(t_2, t_n-t_2) \cdot \sigma(t_2) + ~...\nonumber \\ 
& & + \Delta t_{n-2} R(t_{n-2},t_n-t_{n-2}) \cdot \vec\sigma(t_{n-2}) + 
\Delta t_{n-1} R(t_{n-1},\Delta t_{n-1}) \cdot \vec \sigma(t_{n-1})
\end{eqnarray}


\bibliography{tracer.bib}

\end{document}



