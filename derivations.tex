
\documentclass[12pt]{report}
\usepackage{amsmath}
\usepackage{amssymb}
\usepackage[dvips]{graphicx}

\begin{document}
\begin{flushleft}

\section{Integrating the hydrostatic equation}

\label{hyd}

Suppose first that the temperature varies linearly with the logarithm of
the pressure.  Thus:

\begin{equation}
T = b + k \ln P
\label{hyd:tdep}
\end{equation}

where $T$ is the temperature, $P$ is the pressure and $b$ and $k$ are 
constants.  We wish to integrate the hydrostatic equation:

\begin{equation}
\frac{dP}{dz} = \rho g \linebreak
\label{hyd:hyd}
\end{equation}

where $z$ is altitude, $\rho$ is density and $g$ is the acceleration due to gravity.  Substituting (\ref{hyd:tdep}) into (\ref{hyd:hyd}) and using the ideal gas law allows us to integrate the change in altitude:

\begin{equation}
\Delta z = \int_{P_1}^{P_2} \frac{R(b + k \ln P)}{Pg} dP
\end{equation}

Assuming that both $R$ and $g$ are constant allows us to pull them out of the integration:

\begin{eqnarray}
\Delta z & = & \frac{R}{g} \left [ \int_{P_1}^{P_2} \frac{b}{P} dP + 
\int_{P_1}^{P_2} \frac{k \ln P}{P}dP \right ] \\
 & = & \frac{R}{g} \left [ b \ln P + \frac{k}{2} \ln^2 P \right ]_{P_1}^{P_2} \\
 & = & \frac{R}{g} \left \lbrace b \ln \left ( \frac{P_2}{P_1} \right ) +
 	\frac{k}{2} \left [ \ln^2 P_2 - \ln^2 P_1 \right ] \right \rbrace
\label{hyd:hyd_int}
\end{eqnarray}

The constants, $b$ and $k$, may be calculated from the values of temperature and pressure
at the top and bottom of the layer:

\begin{eqnarray}
k & = & \frac{T_1 - T_2}{\ln (P_2/P_1)} \label{hyd:kdef} \\
b & = & T_1-k \ln P_1
\end{eqnarray}

Substituting this into (\ref{hyd:hyd_int}) and simplifying produces the following result:

\begin{equation}
\Delta z = \frac{R}{g} \left \lbrace T_1 \ln P_2 + T_2 \ln P_1 +
 	\frac{k}{2} \left [ \ln^2 P_2 - \ln^2 P_1 \right ] \right \rbrace
\label{hyd:prev_dz}
\end{equation}

where $k$ is simply taken from (\ref{hyd:kdef}).  Using this integration versus an equivalent
one that assumes a constant temperature throughout does not produce 
noticeably different results when numerically integrating multi-layer profiles.
\linebreak
\linebreak
Equation (\ref{hyd:prev_dz}) actually simplifies further:

\begin{eqnarray}
  \Delta z & = & \frac{R}{g} \left [ \frac {1}{2} T_1 \ln P_1 + \frac{3}{2} T_1 \ln P_2 + 
		\frac{1}{2} T_2 \ln P_1 - \frac{1}{2} T_2 \ln P_2 \right ] 
\end{eqnarray}

***Correction*** 
\linebreak
\linebreak
Equation (\ref{hyd:prev_dz}) should probably be:
\begin{equation}
\Delta z = \frac{R}{g} \left \lbrace T_1 \ln P_2 - T_2 \ln P_1 +
 	\frac{k}{2} \left [ \ln^2 P_2 - \ln^2 P_1 \right ] \right \rbrace
\end{equation}

which simplifies to:

\begin{eqnarray}
  \Delta z & = & \frac{R}{2g} \left [ T_1 \ln P_1 - T_1 \ln P_2 + 
		 T_2 \ln P_1 -  T_2 \ln P_2 \right ] \\
  & = & \frac{R}{2g} (T_1 + T_2) \ln (P_1 / P_2)
\end{eqnarray}

equivalent to the case of an isothermal layer with a temperature equal to
the average of the top and bottom layers.

\section{Ascent rate of a weather balloon}

There are three main forces affecting a rising weather balloon:  $F_b$, 
the bouyancy force, $F_d$, the force of drag and $F_g$, the force of 
gravity.  The bouyancy force acts upwards, the gravitational force
acts downwards, while the drag force counters the direction of motion.
The three forces are given as follows:

\begin{eqnarray}
F_b & = & V \rho_a \\
F_d & = & A \rho_a c_d v^2 \\
F_g & = & m g = V \rho_s g
\end{eqnarray}

where $V$ is the volume of the balloon, $A$ is its cross-sectional area,
$m$ is its mass, $\rho_s$ is its density, $v$ is its speed, 
$\rho_a$ is the density of air and $g$ is the acceleration due to gravity.
Assuming that the balloon is travelling at terminal velocity in an
upward direction, these forces will balance:

\begin{eqnarray}
F_b = F_d + F_g \\
V \rho_a = A \rho_a c_d v^2 + V \rho_s g
\label{asc:fbal2}
\end{eqnarray}

For all axially symetric shapes, the volume $V$ is related to the area $A$ as:

\begin{equation}
V = khA
\end{equation}

where $h$ is the height of the balloon and $k$ is a constant.  For prisms and 
cylinders this constant will be unity: $k=1$.  Substituting this into 
equation (\ref{asc:fbal2}) and solving for $v$ produces the following:

\begin{equation}
v = \sqrt{\frac{k h}{c_d} \left(1 - g \frac{\rho_s}{\rho_a} \right )}
\end{equation}

Since the balloon is composed primarily of gas, if we assume that it takes 
on the ambient temperature, then the ratio between its density
and that of the outside air should be roughly constant:

\begin{equation}
\frac{\rho_s}{\rho_a} \approx \frac{R_{He}}{R_a}
\end{equation}

where $R_a$ is the equivalent gas constant of air and $R_{He}$ is the 
equivalent gas constant for the balloon (likely close to that of Helium--
hence the subscript).  Therefore:

\begin{equation}
v \approx \sqrt{\frac{k h}{c_d} \left(1 - g \frac{R_{He}}{R_a} \right )} \approx const.
\end{equation}

\section{Azimuthal equidistant coordinate system}

The following transformations may be applied to a two-dimensional spherical-
polar coordinate system describing the surface of a shere.  
It's most useful property is that it eliminates the singularity at one pole, where
it is locally Cartesian.  It also defines an azimuthal equidistant projection.

\subsection{Transformations}

\begin{eqnarray}
\tilde x & = & \tilde r \cos \theta \\
\tilde y & = & \tilde r \sin \theta
\end{eqnarray}

where $\theta$ is the longitudinal coordinate in radians and $\tilde r$
 is given as follows:
 
\begin{equation}
\tilde r = \sqrt{x^2 + y^2} = R_e (\pi/2 - \mathfrak h \phi)
\end{equation}

where $R_e$ is the radius of the sphere (planet Earth, typically) and 
$\mathfrak h = \left \lbrace \begin{array}{r} 1 \\ -1 \end{array} \right.$
is the hemisphere for which the transformation is defined.

\subsection{Reverse transformations}

\begin{eqnarray}
\phi & = & \mathfrak h \left ( \frac{\pi}{2} - \frac{\tilde r}{R_e} \right ) \\
\theta & = & \tan^{-1} \left (\frac{\tilde y}{\tilde x} \right )
\end{eqnarray}

\subsection{Transformation between hemispheres}

\begin{eqnarray}
\tilde r_{-\mathfrak h} & = & \pi R - \tilde r_{\mathfrak h} \\
\tilde x_{-\mathfrak h} & = & \frac {\tilde r_{-\mathfrak h} } 
	{\tilde r_{\mathfrak h}} \tilde x_{\mathfrak h} \\
\tilde y_{-\mathfrak h} & = & \frac {\tilde r_{-\mathfrak h} } 
	{\tilde r_{\mathfrak h}} \tilde y_{\mathfrak h}
\end{eqnarray}

where the subscript denotes the hemisphere of the transformation for which the quantity is defined.

\subsection{Velocity transformations}

\begin{eqnarray}
\frac{d \tilde x}{dt} & = & - \mathfrak h v \frac{\tilde x}{\tilde r} - 
		\frac{u \tilde y}{R_e \sin (\tilde r/R_e)} \\
\frac{d \tilde y}{dt} & = & - \mathfrak h u \frac{\tilde x}{\tilde r} - 
		\frac{v \tilde y}{R_e \sin (\tilde r/R_e)}
\end{eqnarray}

where $u$ is the velocity in the $\theta$ direction and $v$ is the velocity in
the $\phi$ direction.  That is:

\begin{eqnarray}
u & = & R_e \cos \phi \frac{d \theta}{dt} \\
v & = & R_e \frac{d \phi}{dt}
\end{eqnarray}

\begin{eqnarray}
u & = & R_e \sin \left (\frac{\tilde r}{R_e} \right ) \frac{\tilde y}{\tilde r^2}
		\left (-\frac{d \tilde x}{dt} \frac{\tilde y}{\tilde x} + 
		\frac{d \tilde y}{dt} \right ) \\
v & = & \frac{1}{\tilde r} \left (-\frac{d \tilde y}{dt} {\tilde x} + 
		\frac{d \tilde x}{dt} \tilde y \right )
\end{eqnarray}

\subsection{Derivatives}

\begin{eqnarray}
\frac{d \theta}{d \tilde x} = \frac{\tilde y}{\tilde r^2} \\
\frac{d \theta}{d \tilde y} = \frac{\tilde x}{\tilde r^2} \\
\frac{d \phi}{d \tilde x} = \frac{\tilde x}{R_e \tilde r} \\
\frac{d \phi}{d \tilde y} = \frac{\tilde y}{R_e \tilde r} \\
\end{eqnarray}

\subsection{Metric}

The metric may be derived from the above derivatives:

\begin{eqnarray}
\left (\frac{ds}{d \tilde x} \right )^2 & = & \frac{1}{\tilde r^2} \left [
	\frac{R_e^2}{\tilde r^2} \sin^2 \left (\frac{\tilde r}{R_e} \right ) 
	\tilde y^2 + \tilde x^2 \right ] \\
\left (\frac{ds}{d \tilde y} \right )^2 & = & \frac{1}{\tilde r^2} \left [
	\frac{R_e^2}{\tilde r^2} \sin^2 \left (\frac{\tilde r}{R_e} \right )
	\tilde x^2 + \tilde y^2 \right ]
\end{eqnarray}

\subsection{Equations of motion}

These have yet to be derived...

\section{Evolution of the tracer gradient}

The equations for a conserved tracer are as follows:

\begin{equation}
\frac{dq}{dt} = \frac{\partial q}{\partial t} + \nabla q \cdot \vec v = 0
\label{tgrad:trac_eq}
\end{equation}

where $q$ is the concentration of the tracer: volume-mixing-ratio for a
compressible flow or density or similar measure for an incompressible one.
Taking the gradient of the middle expression produces the following result:

\begin{equation}
\frac{\partial (\nabla q)}{dt} + \nabla (\nabla q) \cdot \vec v + \nabla q \cdot \nabla \vec v = 0
\label{tgrad:grad_trac}
\end{equation}

As in (\ref{tgrad:trac_eq}), one can apply the full derivative to the tracer gradient, $\nabla q$ in the
same manner as to the tracer itself:

\begin{equation}
\frac{d (\nabla q)}{dt} = \frac{\partial (\nabla q)}{\partial t} + \nabla (\nabla q) \cdot \vec v
\label{tgrad:tgrad_full}
\end{equation}

Combining equations (\ref{tgrad:grad_trac}) and (\ref{tgrad:tgrad_full}) produces
the following exact equation for the evolution of the tracer gradient in Lagrangian
space:

\begin{equation}
\frac{d (\nabla q)}{dt} = - \nabla q \cdot \nabla \vec v 
\end{equation}

This may be represented in the following manner:  we define the matrix $\boldsymbol{H^\prime}$
as follows:

\begin{eqnarray}
\frac{d \boldsymbol{H^\prime}}{dt} & \equiv & \boldsymbol{H^\prime} \cdot \nabla \vec v \\
\boldsymbol{H^\prime_0} \equiv \boldsymbol{H^\prime}(t=0) & = & \boldsymbol{I}\nonumber
\label{tgrad:hprime_def}
\end{eqnarray}

It follows that:

\begin{eqnarray}
\nabla q  & = & \nabla q_0 \cdot \boldsymbol{H^\prime} \label{tgrad:qevol}\\
q(t=0) & \equiv & q_0\nonumber
\end{eqnarray}

These definitions are parallel to those describing the evolution of a minute 
error vector, $\delta \vec x$, derived for the purposes of computing the
Lyapunov exponents.  Thus:

\begin{eqnarray}
\frac{d (\delta \vec x)}{dt} & = & \nabla \vec v \cdot \delta \vec x \label{tgrad:delx_evol}\\
\frac{d \boldsymbol H}{dt} & = & \nabla \vec v \cdot \boldsymbol H; ~~
 \boldsymbol {H_0} \equiv \boldsymbol{I} \label{tgrad:hdef}\\
\delta \vec x & = & \boldsymbol{H} \cdot \delta \vec x_0
\end{eqnarray}

The difference is that these equations, being derived from a Taylor expansion, are 
only approximate for error vectors of non-negligible magnitude.  It is easy to show that:

\begin{equation}
\boldsymbol{H^\prime} = \boldsymbol{H^{-1}}
\label{tgrad:hprimevh}
\end{equation}

We will use two methods to prove (\ref{tgrad:hprimevh}): one more hand-waving
that calls upon the properties of advected tracers, and another more rigorous that
applies to all equations of the form of (\ref{tgrad:hprime_def}) and (\ref{tgrad:hdef}).
\linebreak
\linebreak
Consider a minute tracer difference:
\begin{equation}
\delta q = \delta \vec x \cdot \nabla q
\end{equation}
defined by the difference 
in tracer concentration at two fixed locations in Lagrangian space.  It is
easy to see that this quantity should be conserved, moreover it can also 
be shown from equations (\ref{tgrad:qevol}) and (\ref{tgrad:delx_evol}):

\begin{eqnarray}
\frac{d (\delta q)}{dt} & = & \nabla q \cdot \frac{d (\delta \vec x)}{dt} \cdot + 
		\frac{d (\nabla q)}{dt} \cdot \delta \vec x \\
 & = & \nabla q \cdot \nabla \vec v \cdot \delta \vec x - \nabla q \cdot \nabla \vec v \cdot \delta \vec x \\
 & = & 0
\end{eqnarray}

To prove equation (\ref{tgrad:hprimevh}):

\begin{eqnarray}
\nabla q & = & \delta q (\delta \vec x)^{-1} \\
\nabla q_0 \cdot \boldsymbol{H^\prime} & = & \delta q (\delta \vec x)^{-1} \cdot \boldsymbol{H^{-1}}\\
\boldsymbol{H^\prime} & = & \boldsymbol{H^{-1}}
\end{eqnarray}

As a more rigorous and general approach, consider the time evolution of
the quantity, $\boldsymbol{H^\prime} \cdot \boldsymbol{H}$:

\begin{eqnarray}
\frac{d (\boldsymbol{H^\prime} \cdot \boldsymbol{H})}{dt} & = & 
	\boldsymbol{H^\prime} \cdot \nabla \vec v \cdot \boldsymbol{H} -
	\boldsymbol{H^\prime} \cdot \nabla \vec v \cdot \boldsymbol{H} \\
 & = & 0 \nonumber
\end{eqnarray}

If both matrices are initialised at the same time with the identity matrix, then:

\begin{equation}
\boldsymbol{H^\prime} \cdot \boldsymbol{H} = \boldsymbol{H^\prime_0} \cdot \boldsymbol{H_0} = \boldsymbol I
\end{equation}

\section{Adaptive Gaussian filters}

\subsection{Extrapolation of the confidence rating}

Consider a joint probability composed of two equal-sized Gaussians of unit width.
The coordinate system is defined so that the class-border lies on the origin.  The
difference between the two conditional probabilities is given as:

\begin{eqnarray}
R(x) & = & P(1 | x) - P(2 | x) \\
 & = & \frac{P(1, x) - P(2, x)}{P(1, x) + P(2, x)} \\
 & = & \frac{e^{-\frac{(x-b)^2}{2\sigma^2}} - e^{-\frac{(x+b)^2}{2\sigma^2}}}
 		{e^{-\frac{(x-b)^2}{2\sigma^2}} + e^{-\frac{(x+b)^2}{2\sigma^2}}}
\end{eqnarray}
 
 where $b$ is the offset of each of the classes and $\sigma$ is their width.
 Simplifying:
 
\begin{eqnarray}
R(x) & = & \frac{e^{-\frac{x^2 - 2bx + b^2}{2\sigma}} - e^{-\frac{x^2 + 2bx + b^2}{2\sigma}}}
		{e^{-\frac{x^2 - 2bx + b^2}{2\sigma}} + e^{-\frac{x^2 + 2bx + b^2}{2\sigma}}} \\
& = & \frac{e^{-\frac{bx}{\sigma^2}} - e^{\frac{bx}{\sigma^2}}}
		{e^{-\frac{bx}{\sigma^2}} + e^{\frac{bx}{\sigma^2}}} \\
& = & \tanh(bx/\sigma^2)
\end{eqnarray}

We wish to constrain the $R(x)$ so that it has a unit slope at the origin:

\begin{eqnarray}
\frac{dR}{dt} & = & \frac{b}{\sigma^2} \mathrm{sech}^2 \left (\frac{bx}{\sigma^2} \right ) = 1 \\
\frac{b}{\sigma^2} & = & 1
\end{eqnarray}

or,

\begin{equation}
R(x) = \tanh(x)
\label{cext:r1sim}
\end{equation}

Now consider a pair of classes in multiple dimensions.  The difference between
the conditional probabilities is given, as before, by:

\begin{equation}
R(\vec x) = P(1 | \vec x) - P(2 | \vec x)
\end{equation}
  
  The location
of a single point on the class border is known:

\begin{equation}
R(\vec x_b)=0
\end{equation}

where $\vec x_b$ is the border point.  Its gradient vector, $\nabla R(\vec x_b)$
is also known.  Now consider an arbitrary point, $\vec x$.  We wish to extrapolate
$R(\vec x)$ to that point.  Consider a transformed coordinate defined by 
projecting the difference vector, $\vec x - \vec x_b$, onto the gradient vector
and multiplying by the magnitude of the latter:

\begin{eqnarray}
p & = & |\vec \nabla R(\vec x_b)| \underset{\vec \nabla R(\vec x_b)} {\mathrm{proj}} (\vec x - \vec x_b) \\
  & = & \nabla R(\vec x_b) \cdot (\vec x - \vec x_b)
\end{eqnarray}

We can approximate $R(\vec x)$ as in equation (\ref{cext:r1sim}) by:

\begin{equation}
R(\vec x) \approx \tilde R(\vec x) = \tanh(p)
\end{equation}

It is easy to show:

\begin{equation}
\nabla \tilde R(\vec x_b) = \nabla R(\vec x_b)
\end{equation}

and:

\begin{equation}
\tilde R(\vec x_b) = 0
\end{equation}

\subsection{Relationship between $k$ and $W_c$}

We assume that the value of the pdf, $P(\vec x)$ is more or less constant within
the immediate vicinity.  From its definition:

\begin{equation}
P(\vec x) \approx \frac{k}{n\delta V}
\end{equation}

where $k$ is the number of samples within a volume element $\delta V$ and $n$ is the
total number of samples.  The approximate the distance of the farthest sample, $d$,
as follows:

\begin{eqnarray}
d & \approx & (\delta V)^{1/D} \\
	& \approx & \frac{k^{1/D}} {P^{1/D} n^{1/D}}
\label{kvwc:pknn}
\end{eqnarray}

where $D$ is the number of dimensions.  Assuming the value of $P$ is approximated via
Gaussian filtering:

\begin{equation}
P(\vec x) \approx \frac{W}{(2\pi)^{D/2}\sigma^{D}}
\label{kvwc:pgauss}
\end{equation}

Substituting (\ref{kvwc:pgauss}) into (\ref{kvwc:pknn}) produces the following result:

\begin{equation}
d \approx \frac{\sqrt{2\pi} \sigma k^{1/D}}{W^{1/D}}
\end{equation}

The approximate ratio of the largest possible weight wrt the smallest
is given as:

\begin{equation}
f \approx \exp \left (- \frac{\pi k^{2/D}}{W^{2/D}} \right )
\end{equation}

Solving for $k$:

\begin{equation}
k \approx \left (- \frac{\ln f}{\pi} \right )^{D/2} W
\end{equation}

To ensure that the actual ratio of the smallest weight wrt the largest is less than
or equal to $f$, $k$ should be chosen to be somewhat greater than the rhs.
		
\subsection{Better relationship between $k$ and $W_c$}

From the definition of the probability density:
\begin{equation}
P \approx \frac{k}{nV}
\end{equation}
where $k$ is the number of samples in the estimate,
$n$ is the total number of samples and $V$ is volume.
(The equation is exact in the limit as $n \rightarrow \infty$
and $V \rightarrow 0$.)
The volume of a hypersphere of radius $r$ and dimension $D$ is:
\begin{equation}
V = \frac{\pi^{D/2} r^D}{\Gamma(D/2+1)}
\end{equation}
giving us:
\begin{equation}
P \approx \frac{k \Gamma(D/2 + 1)}{n \pi^{D/2} r^D}
\end{equation}
Solving for $r$:
\begin{equation}
r \approx \frac{1}{\sqrt{\pi}}\left [\frac{k \Gamma(D/2+1)}{nP} \right]^{1/D}
\end{equation}
Substituting the equation of $P$ that has been approximated by
AGF:
\begin{equation}
P \approx \frac{W}{n(2\pi)^{D/2}\sigma}
\end{equation}
we get:
\begin{equation}
r \approx \sqrt{2} \sigma \left [\frac{k \Gamma(D/2 + 1)}{W} \right ]^{1/D}
\end{equation}
Finally, we substitute this into the filter function to
get the smallest value for the weight:
\begin{eqnarray}
f & = & \exp \left (-\frac{r^2}{2\sigma^2} \right ) \\
& = & \exp \left \lbrace -\left [ \frac{k \Gamma(D/2+1)}{W} \right ]^{2/D} \right \rbrace
\end{eqnarray}
Solving for $k$:
\begin{equation}
k \approx \frac{(-\ln f)^{D/2} W}{\Gamma(D/2 + 1)}
\end{equation}

This will generate much more optimistic estimates for $k$ than the
previous derivation.

\subsection{Gradient of an AGF estimate}

Since the final result is typically just a summation of the filter weights, 
usually multiplied by some nominally constant values, we start by taking their gradients: 

\begin{equation}
  \frac{\partial w_i}{\partial x_j} = \left (\frac{\partial w_i}{\partial x_j} \right )_\sigma
		+ \frac{\partial w_i}{\partial \sigma} \frac{\partial \sigma}{\partial x_j}
  \label{gradagf:grad_w}
\end{equation}

Since the filter width is not constant, we must include a term to account for its change.
The second factor of the second term, the gradient of the filter width, is not known,
but we can easily solve for it by taking the derivative of the total weight, which is
a constant:

\begin{eqnarray}
  \frac{\partial W}{\partial x_j} & = & \frac{\partial}{\partial x_j} \sum_i w_i \\
  & = & \sum_i \frac{\partial w_i}{\partial x_j} \\
  & = & \sum_i \left [\left (\frac{\partial w_i}{\partial x_j} \right )_\sigma
                + \frac{\partial w_i}{\partial \sigma} 
		\frac{\partial \sigma}{\partial x_j} \right ] \\
  & = & 0
\end{eqnarray}

Solving:

\begin{equation}
  \frac{\partial \sigma}{\partial x_j} = - \frac
		{\left (\frac{\partial w_i}{\partial x_j} \right )_\sigma}
		{\frac{\partial w_i}{\partial \sigma}}
  \label{gradagf:grad_sigma}
\end{equation}

For the case of a Gaussian filter the weights are given by:

\begin{eqnarray}
  w_i & = & \exp \left [ - \frac{d_i^2}{2 \sigma^2} \right ] \\
  	& = & \exp \left [ - \frac{\sum_j (x_{ij} - x_j)^2}{2 \sigma^2} \right ]
\end{eqnarray}

which generates the following partials:

\begin{eqnarray}
  \left (\frac{\partial w_i}{\partial x_j} \right )_\sigma & = &
		 \frac{(x_{ij} - x_j)}{\sigma^2} w_i \\
  {\frac{\partial w_i}{\partial \sigma}} & = & \frac{d_i^2}{\sigma^3} w_i
\end{eqnarray}

Subsituting these into equations (\ref{gradagf:grad_sigma}) and 
(\ref{gradagf:grad_w}) respectively, produces the following:

\begin{eqnarray}
  \frac{\partial \sigma}{\partial x_j} & = & - \frac{\sigma \sum_i (x_{ij} - x_j) w_i}
		{\sum_i d_i^2 w_i} \\
  \frac{\partial w_i}{\partial x_j} & = & \frac{w_i}{\sigma^2} \left [x_{ij} - x_j 
		- d_i^2 \frac{\sum_k (x_{kj} - x_j) w_k} {\sum_k d_k^2 w_k} \right ]
\end{eqnarray}

To calculate the gradient of an estimate, we simply multiply the sample ordinates
with the gradients of the weights and divide by the total weight (which is constant):

\begin{eqnarray}
  \frac{\partial \bar y}{\partial x_j} & = & \sum_i \frac{\partial w_i}{\partial x_j} y_i \\
  & = & \frac{1}{W \sigma^2} \sum_i w_i y_i \left [ x_{ij} - x_j
                - d_i^2 \frac{\sum_k (x_{kj} - x_j) w_k} {\sum_k d_k^2 w_k} \right ]
\end{eqnarray}

For a two class classification, the gradient of the difference of the conditional 
probabilities is given as follows:

\begin{eqnarray}
  \frac{\partial R}{\partial x_j} & \approx & \sum_i \frac{\partial w_i}{\partial x_j} (2c_i-3) \\
  & \approx & \frac{1}{W \sigma^2} \sum_i w_i (2c_i - 3) \left [ x_{ij} - x_j
                - d_i^2 \frac{\sigma \sum_k (x_{kj} - x_j) w_k} {\sum_k d_k^2 w_k} \right ]
\end{eqnarray}

using our rather awkward convention of enumerating the first class by ``$1$'' 
and the second by ``$2$.'' \linebreak
\linebreak
The derivatives of a probability density estimate are:

\begin{eqnarray}
  \frac{\partial P (\vec x)}{\partial x_j} & \approx & 
		\frac{1}{nN} \left (\frac{\partial W}{\partial x_j} - 
		\frac{W}{N} \frac{\partial N} {\partial x_j} \right ) \\
  & \approx & \frac {W}{n \sigma} \frac{\partial} {\partial x_j} \\
  & \approx & \frac{\sum_i (x_{ij} - x_j) w_i} {n \sum_i d_i^2 w_i}
\end{eqnarray}

which may be useful both for clustering analysis and
calculating an error tolerance if I ever get around to these things...\linebreak
\linebreak
Definitions of all the symbols can be found in the paper, ``Adaptive Guassian filters:
a powerful new method for supervised learning.''

\end{flushleft}
\end{document}

